\section{ИНТЕРПРЕТАЦИЯ РЕЗУЛЬТАТОВ, ВЫВОДЫ}

При большой вычислительной сложности задачи (при $n=1100,\, 2500,\, 5000,\, 10000$) ускорение растет при увеличении числа процессов, при этом, чем больше вычислительная сложность, тем быстрее растет ускорение, что видно из Рис. \ref{fig:speedup_vs_threads}. Закон Амдала при большой вычислительной сложности выполняется, при этом можно предположить, что доля последовательного кода уменьшается с увеличением вычислительной сложности задачи.

При малой вычислительной сложности (при $n=100$) ускорение сильно увеличивается в случае двух потоков, при этом преодолевается теоретический предел ускорения (ускорение равно числу потоков), что видно из Рис. \ref{fig:speedup_vs_threads}. Таким образом, при малой вычислительной сложности закон Амдала нарушается. При дальнейшем увеличении числа потоков ускорение уменьшается, что можно связать с тем, что при малой сложности задачи на обмен сообщениями между процессами тратится число операций, сравнимое с числом арифметических операций в задаче.

В сравнении с оптимизацией с использованием OpenMP оптимизация с использованием MPI хуже ускоряет алгоритм при большой вычислительной сложности задачи (при $n=1100,\, 2500$; в прошлой лабораторной работе рассматривались лишь $n=100,\, 1100,\, 2500$, поэтому здесь рассматриваются только такие $n$). Так, если на 6 потоках при $n=2500$ оптимизация с помощью OpenMP позволяет достичь ускорения, примерно равного 5, то оптимизация с использованием MPI позволяет достичь ускорения, примерного равного 3. При малой вычислительной сложности проводить сравнение двух оптимизаций некорректно, так как в прошлой лабораторной работе при $n=100$ ускорение было измерено с большой погрешностью, что не позволяет сделать какие-либо выводы о том, насколько хорошо оптимизирован алгоритм.