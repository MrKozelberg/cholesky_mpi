\section{ОПИСАНИЕ ПОСЛЕДОВАТЕЛЬНОГО АЛГОРИТМА РЕШЕНИЯ ЗАДАЧИ}

Решение задачи разбивается на два этапа: 
\begin{enumerate}
    \item Применение алгоритма Холецкого к симметричной и положительно определенной матрице $\vb{A}$, то есть ее представление в виде $\vb{A} = \vb{L}\vb{L}^T$, где матрица $\vb{L}$ — нижняя треугольная матрица со строго положительными элементами на диагонали;
    \item Последовательное решение двух СЛАУ с треугольными матрицами $\vb{L}\vb{Y} = \vb{B}$, откуда нужно найти неизвестную матрицу $\vb{Y}$, и $\vb{L}^T\vb{X} = \vb{Y}$, где на основе найденной из предыдущей СЛАУ матрицы $\vb{Y}$ нужно найти матрицу $\vb{X}$.
\end{enumerate}

Рассмотрим подробно каждый этап.

\subsection{РАЗЛОЖЕНИЕ ХОЛЕЦКОГО}
\label{sec:cholesky}

В данной работе используется блочная версия разложения Холецкого, так как такая версия алгоритма лучше параллелится (см. \href{https://www.cs.utexas.edu/~flame/Notes/NotesOnCholReal.pdf}{работу}). Ширина блока задается числом $n_b$, а ширина матрицы $\vb{A}$ --- числом $n$. Используемый вариант блочного алгоритма Холецкого может быть описан следующим образом:
\begin{enumerate}
    \item Матрица $\vb{A}$ представляется в виде $\mqty(\vb{A}_{11} & \star \\ \vb{A}_{21} & \vb{A}_{22})$, где $\vb{A}_{11}$ --- матрица $b \times b$ (для удобства рассматриваем случай, когда $b$ укладывается целое число раз в $n$);
    \item Вычисляется матрица $\vb{L}_{11}$ такая, что $\vb{L}_{11} \vb{L}^T_{11} = \vb{A}_{11}$, то есть матрица $\vb{L}_{11}$ --- фактор Холецкого матрицы $\vb{A}_{11}$, и матрица $\vb{A}_{11}$ замещается $\vb{L}_{11}$, то есть $\vb{A}_{11} := \vb{L}_{11}$;
    \item Вычисляется матрица $\vb{L}_{21} = \vb{A}_{21} \qty(\vb{L}_{11})^{-1}$, которая замещает матрицу $\vb{A}_{21}$, то есть $\vb{A}_{21} := \vb{L}_{21}$;
    \item Обновляется матрица $\vb{A}_{22}$ так, что $\vb{A}_{22} := \vb{A}_{22} - \vb{L}_{21} \vb{L}^T_{21}$;
    \item Алгоритм снова начинается с шага 1 с матрицей $\vb{A} = \vb{A}_{22}$ (если матрица $\vb{A}_{22}$ не пустая).
\end{enumerate}
В Листинге \ref{listing:cholesky_serial} представлена реализация на современном Фортран последовательного блочного алгоритма разложения Холецкого, который был описан выше.

\subsection{ОБРАТНАЯ ПОДСТАНОВКА}
Вторым этапом решения задачи является обратная подстановка. То есть последовательно решаются 2 СЛАУ с треугольными матрицами
$$
\begin{cases}
    \vb{L}\vb{y} = \vb{b}^{(i)},\\
    \vb{L}^T\vb{x} = \vb{y}
\end{cases}
$$
для $i=1,2,\ldots,10$. Вектор $\vb{y}$ находится по формулам
$$
y_1 = \dfrac{b^{(i)}_1}{l_{11}};\; y_{j} = \dfrac{b_j^{(i)} - \sum_{p=1}^{j-1} l_{jp} y_p}{l_{jj}},\; j=2,3,\ldots,n,
$$
где $n$ --- число столбцов матрицы $\vb{A}$. Вектор $\vb{x}$ находится аналогично
$$
x_n = \dfrac{y_n}{{l^T}_{nn}};\; x_{j} = \dfrac{y_j - \sum_{p=j+1}^{n} {l^T}_{jp} x_p}{{l^T}_{jj}},\; j=n-1,n-2,\ldots,1.
$$
Последовательный вариант реализации второго этапа решения задачи представлен в Листинге \ref{listing:backward_substitution}.